\chapter*{Introduction}

1er draft... Puis en fonction de l'intro du chapitre 2, il risque d'y avoir quelques redites.

\paragraph{}
Dans un ordinateur, le processeur accède aux instructions du programme à exécuter ainsi qu'aux données nécessaires
à son exécution depuis la mémoire.
De ce fait, la mémoire est un élément déterminant dans ses performances. \\
Les registres du processeur sont les emplacements de mémoire les plus rapides car directement implantées sur le processeur, mais 
ils sont aussi les plus chers notamment pour des raisons évidentes de place sur le processeur. Leur taille est donc limitée à quelques dizaines d'octets. \\
Pour pallier ce peu d'espace mémoire, il existe une hiérarchie de la mémoire. Plus on s'éloigne du processeur, plus les mémoires sont grandes et peu chères.
En contrepartie, leur vitesse diminue : l'accès à ces mémoires nécéssitent de plus en plus de cycles processeur.
\paragraph{}
Dans le cadre de ce PFA, nous nous intéressons à la mémoire cache. Dans la hiérarchie, elle se situe en dessous des registres du processeur.
Tout comme les registres, elle est interne au processeur. Elle est toutefois bien moins chère et plus large que cette dernière. En contrepartie, son temps d'accès est supérieur.
L'amélioration des performances d'un processeur, outre l'augmentation du nombre de coeurs ou de sa fréquence d'horloge peut provenir également de
l'optimisation de sa mémoire cache. C'est en ce sens que nous développons le simulateur de caches Cassis. \\
Le principe du simulateur est de rejouer une trace d'exécution d'un programme multithreadé, préalablement
généré, afin d'obtenir des métriques associées aux caches sur une architecture de processeur paramétrable. Si les métriques obtenues par Cassis sont relativement peu utiles pour l'utilisateur lambda,
elles peuvent néanmoins être très utiles pour la recherche, d'autant plus qu'il existe actuellement peu d'outils similaires.
\paragraph{}
Prérequis pour réaliser ce simulateur, il faut connaître le fonctionnement des caches. C'est ce que nous verrons dans la première partie de ce document.
Dès lors nous pourrons examiner plus en détail le fonctionnement de Cassis.
