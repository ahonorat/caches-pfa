\documentclass{beamer}
\usepackage[utf8]{inputenc}
\usepackage[T1]{fontenc}
\usepackage[french]{babel}

\usepackage{amsmath}
\usepackage{amssymb}
\usepackage{color}
\usepackage{graphicx}

\usepackage{listings}
\lstset{ %
  backgroundcolor=\color{white},   % choose the background color; you must add \usepackage{color} or \usepackage{xcolor}
  basicstyle=\tiny,        % the size of the fonts that are used for the code
  breakatwhitespace=false,         % sets if automatic breaks should only happen at whitespace
  breaklines=true,                 % sets automatic line breaking
  captionpos=b,                    % sets the caption-position to bottom
  commentstyle=\color{green},      % comment style
  deletekeywords={...},            % if you want to delete keywords from the given language
  escapeinside={\%*}{*)},          % if you want to add LaTeX within your code
  extendedchars=true,              % lets you use non-ASCII characters; for 8-bits encodings only, does not work with UTF-8
  frame=single,                    % adds a frame around the code
  keepspaces=true,                 % keeps spaces in text, useful for keeping indentation of code (possibly needs columns=flexible)
  keywordstyle=\color{blue},       % keyword style
  language=xml,                    % the language of the code
  morekeywords={Level,Cache,Architecture,...},            % if you want to add more keywords to the set
  numbers=left,                    % where to put the line-numbers; possible values are (none, left, right)
  numbersep=5pt,                   % how far the line-numbers are from the code
  numberstyle=\tiny\color{blue},  % the style that is used for the line-numbers
  rulecolor=\color{black},         % if not set, the frame-color may be changed on line-breaks within not-black text (e.g. comments (green here))
  showspaces=false,                % show spaces everywhere adding particular underscores; it overrides 'showstringspaces'
  showstringspaces=false,          % underline spaces within strings only
  showtabs=false,                  % show tabs within strings adding particular underscores
  stepnumber=2,                    % the step between two line-numbers. If it's 1, each line will be numbered
  stringstyle=\color{red},         % string literal style
  tabsize=2,                       % sets default tabsize to 2 spaces
  title=\lstname                   % show the filename of files included with \lstinputlisting; also try caption instead of title
}

\usetheme{CambridgeUS}
\title[PFA - Simulateur de caches multi-c\oe ur]{PFA - Simulateur de caches multi-c\oe ur}
\author[]{Nicolas Dubois, Pierre Goudet, Nicolas Heng,\\Alexandre Honorat, Gilles Marait, Grégoire Pichon}
\institute[ENSEIRB-MATMECA]{ENSEIRB-MATMECA}
\date{\today}
%\date{}

\setbeamercolor{title}{bg=red!65!black,fg=white}

\begin{document}

\setlength{\unitlength}{1cm}

\begin{frame}{Présentation}

\titlepage

\end{frame}

%\AtBeginSection[]
%{
%\begin{frame}<beamer>
%  \frametitle{Plan}
%  \tableofcontents[currentsection]
%\end{frame}
%}

\section*{Fonctionnalités}
\begin{frame}
\begin{block}{Fonctionnalités demandées}
\begin{itemize}
  \item{Rejouer et analyser une trace de MAQAO}
  \item{Pouvoir paramétrer l'architecture simulée (HWLoc)}
  \item{Choisir les politiques de gestion des caches à utiliser}
  \item{Gérer les l'entrelacement des threads}
  \item{Générer des données (statistiques) sur l'exécution de la trace}
  \item{UNIX compatible uniquement (MAQAO)}
\end{itemize}
\end{block}
\end{frame}

\begin{frame}
\begin{block}{Fonctionnalités optionnelles}
\begin{itemize}
  \item{Ajouter d'autres métriques (suivre des plages d'adresses)}
  \item{Optimiser le simulateur (boucles)}
  \item{Utiliser des graphes dans le code}
  \item{Ajouter des politiques globales}
  \item{Fournir plusieurs types de sorties pour les statistiques}
\end{itemize}
\end{block}
\end{frame}

\section{Détails techniques}
\begin{frame}
\begin{block}{Paramètres}
\begin{itemize}
  \item{L'architecture simulée dans fichier XML, avec les paramètres disponibles}
  \item{L'entrelacement des threads dans un fichier LUA}
  \item{Les statistiques demandées}
\end{itemize}
\end{block}
\end{frame}

\begin{frame}
\begin{block}{Politiques de remplacement simulées}
\begin{itemize}
  \item{LRU}
  \item{LFU}
  \item{FIFO}
\end{itemize}
\end{block}

\begin{block}{Politiques de cohérence simulées}
\begin{itemize}
  \item{MSI}
  \item{MESI}
  \item{MOSI}
  \item{MOESI}
\end{itemize}
\end{block}

\begin{block}{Politiques globales simulées}
\begin{itemize}
  \item{LLC inclusif}
  \item{Snooping à tous les niveaux}
  \item{Utilisation d'un \emph{directory manager}}
\end{itemize}
\end{block}

\end{frame}


\begin{frame}[fragile]
Exemple de fichier de configuration de l'architecture
\begin{lstlisting}
<?xml version="1.0"?>
<Architecture name="x86_64" CPU_name="Intel(R) Core(TM) i5-3340M CPU @ 2.70GHz (modified)" number_levels="3">
    <Level depth="3" coherence_protocol="MESI" is_inclusive="true"/>
    <Level depth="2" coherence_protocol="MESI" is_inclusive="true"/>
    <Level depth="1" coherence_protocol="MESI" is_inclusive="true"/>
    <Cache depth="3" cache_size="3145728" cache_linesize="64" cache_associativity="12" replacement_protocol="LRU">
        <Cache depth="2" cache_size="262144" cache_linesize="64" cache_associativity="8" replacement_protocol="LRU">
            <Cache depth="1" cache_size="32768" cache_linesize="64" cache_associativity="8" replacement_protocol="LRU"/>
            <Cache depth="1" cache_size="32768" cache_linesize="64" cache_associativity="8" replacement_protocol="LRU"/>
        </Cache>
        <Cache depth="2" cache_size="262144" cache_linesize="64" cache_associativity="8" replacement_protocol="LRU">
            <Cache depth="1" cache_size="32768" cache_linesize="64" cache_associativity="8" replacement_protocol="LRU"/>
            <Cache depth="1" cache_size="32768" cache_linesize="64" cache_associativity="8" replacement_protocol="LRU"/>
        </Cache>
    </Cache>
</Architecture>
\end{lstlisting}
\end{frame}

\begin{frame}
\begin{block}{Métriques}
\begin{itemize}
  \item{Cache hits}
  \item{Cache misses}
  \item{Write-backs}
  \item{Broadcasts}
\end{itemize}
\end{block}
\end{frame}

\subsection{Sortie}
\begin{frame}[fragile]
\begin{verbatim}
Liste des caches
L3 (misses:  9, hits: 12, writes_back: 0, broadcasts: 36)
L2 (misses: 11, hits:  9, writes_back: 8, broadcasts: 20)
L2 (misses: 10, hits: 10, writes_back: 0, broadcasts: 20)
L1 (misses: 10, hits:  3, writes_back: 7, broadcasts: 10)
L1 (misses: 10, hits:  3, writes_back: 7, broadcasts: 10)
L1 (misses: 10, hits:  3, writes_back: 7, broadcasts: 10)
L1 (misses: 10, hits:  3, writes_back: 0, broadcasts: 10)
\end{verbatim}
\end{frame}

\section*{Fin}
\begin{frame}
\begin{center}
Merci de votre attention.
\end{center}
\end{frame}

\end{document}
