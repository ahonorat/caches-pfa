\section{Description générale du produit}

\paragraph{}
Ce projet consiste en la création d'un simulateur de caches pour une architecture multi-c\oe ur (exclusivement \verb!x64!). La simulations des caches revient à parcourir des fichiers de traces d'exécutions d'un ou plusieurs programmes, afin de rejouer l'ensemble des accès et modifications de données effectuées par les processeurs et leurs caches associés lorsqu'ils ont réellement éxécuté ces programmes. Le résultat produit par le simulateur consiste en une analyse du jeu d'exécution, notamment par le nombre de \emph{cache-miss} pour chaque cache.

\subsection{Environnement de fonctionnement}

\paragraph{Contraintes d'interfaçage}
Habituellement plusieurs outils sont utilisés pour l'étude d'une exécution parallèle, et les résultats de deux de ces outils (\texttt{MAQAO} et \texttt{HWLoc}) sont nécessaires au simulateur de caches. En effet \texttt{MAQAO} génère les traces d'exécutions à analyser, tandis que \texttt{HWLoc}
génère les fichiers de configurtion de l'architecture des caches. Bien sûr il sera possible d'utiliser le simulateur en dehors de ces outils, mais la forme des fichiers devra respecter leurs normes. 

\paragraph{Système d'exploitation compatible}
Le simulateur sera uniquement développé pour les systèmes basés sur Linux. Bien qu'écrit en \texttt{C}, la lecture de l'emplacement des fichiers de configuration et l'installation automatique (par un Makefile) ne gérera par Windows. Par ailleurs bien que les traces portent obligatoirement sur des adresses codées sur 64 bits, il sera possible d'exécuter le simulateur sur une architecture \verb!x86! (mais cela risque de nuire aux performances du simulateur).

\paragraph{Visualisation des résultats}
La visualisation des résultats (de même que le lancement du simulateur) se fera dans une console Linux. Aucune interface graphique n'est prévue.

\subsection{Fonctionnalités nominales}
Nominalement le simulateur propose les fonctionnalités suivantes :
\begin{itemize}
\item{configuration de \emph{n} niveaux de caches, chaque niveau ayant un nombre de de caches donnés ainsi qu'une taille fixées par l'utilisateur, compatible avec \texttt{HWLoc},}
\item{configuration de la politique de cohérence pour chaque niveau de cache,}
\item{configuration de la politique de remplacement pour chaque (niveau de ?) cache,}
\item{configuration de la politique de recherche parmi les niveaux de caches,}
\item{configuration de la politique de gestion des threads,}
\item{lecture et simulation d'une trace par c\oe ur au format compressé (générée par \texttt{MAQAO}),}
\item{comptage du nombre de \emph{cache-miss} par cache et par thread.}
\end{itemize}


\subsection{Fonctionnalités annexes}
Si les délais le permettent, plusieurs options seront rajoutées :
\begin{itemize}
\item{optimisation des boucles d'instructions (ne rejoue que la partie non cyclique de la boucle),}
\item{implémentation de métriques supplémentaires (nombre de \emph{cache-miss} par instruction, instruction concurrente responsable du \emph{cache-miss}, etc\ldots,}
\item{(configuration de la politique de remplacement pour chaque cache,)}
\item{gestion avancée des threads, sans l'obligation d'avoir une trace par c\oe ur.}
\end{itemize}
