L'objectif principal du projet, à savoir réaliser un simulateur de caches facilement configurable, a été réalisé. Que cela soit sur la nature des caches (type, taille), ou sur les politiques de cohérence et encore de remplacement, un utilisateur peut facilement tester un programme avec uniquement des changements simples sur les fichiers, souvent sans nécessiter de recompilation.

En ce qui concerne les politiques de cohérence, elles sont très facilement configurables par l'utilisateur, qui peut en ajouter sans rentrer dans le code. En revanche les politiques de remplacement nécessitent de comprendre un peu le code associé, afin de réaliser aux bon endroits certaines actions. L'entrelacement des threads est facilement configurable en lua. Pour la partie sur les types de caches (inclusif, exclusif, non-inclusif), il paraît difficile d'en ajouter, mais à notre connaissance aucun autre type de cache est existant.\\

Le logiciel peut être utilisé pour comparer les résultats de deux programmes, et voir lequel semble le plus efficace relativement aux caches. Pour fournir des statistiques plus proches de la réalité, il faudrait soit calibrer la sortie du logiciel en fonction de benchmarks connues -- ce qui empecherait de simuler des architectures inexistantes --, soit utiliser le simulateur en complément d'autres outils se focalisant plus sur des études de performance.\\

Il serait imaginable d'adapter \textsf{Cassis} aux architectures hétérogènes ou aux architectures NUMA (\emph{Non Uniforme Memory Access}). Cependant, au delà des modifications à effectuer au niveau du code, il y aurait d'autres politiques et beaucoup plus de contraintes à intégrer. Une autre voie d'amélioration est la modélisation de la bande-passante, qui nécessiterait aussi beaucoup d'adaptations mais apporterait un gain considérable pour l'étude des politiques telles que MOESI.\\

Au-delà des objectifs scolaires, l'équipe de \textsf{Cassis} espère que ce travail pourra être repris et amélioré, surtout si la version actuelle ne répondrait pas bien aux attentes des potentiels utilisateurs. La documentation réalisée et la forte paramétrisation possible ont pour vocation de permettre ces améliorations.


