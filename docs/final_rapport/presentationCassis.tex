
En vrac, non trié.

===> Rajouter une intro de partie rappelant la nécéssité d'un simulateur ? (par manque de connaissance/documentation dans le domaine, etc...)

===> Enchaîner ensuite sur le fait qu'on doit pouvoir choisir les paramètres de simulation (pour pouvoir simuler des architectures variés, et même expérimentaux ou pour obtenir le suivi de certaines données particulières)

===> Comment on a implémenté cette entre-guillemets "modularité" -- comment l'utilisateur peut choisir ses paramètres (script Lua pour l'entrelacement, XML c'est déjà fait, etc...)

===> D'autres trucs sur l'implém, par exemple préciser ce que ne fait pas le simulateur (calcul de bande passante, prise en compte du prefetching) ???

===> Exemple d'entrée/sortie, comparaison de la sortie avec ceux obtenus avec d'autres logiciels (compteur hard) + graphes en fonction de la taille de l'entrée (pour voir s'il n'y a pas de déviation lorsqu'on a des programmes plus complexes)



\section{Cadre du simulateur}

\subsection{Origine du projet}

\subsection{Outils à disposition de l'utilisateur}

\subsection{Données simulées : analyses possibles des résultats}


\section{Déroulement de la simulation}

\subsection{Traitement d'une instruction : load/store}

\subsection{Rapatriement prédictif d'une ligne}

\subsection{Mise à jour des lignes}

\subsection{Problème d'ajout de ligne dans un cache plein}


\section{Validation de la simulation}

\subsection{Tests unitaires}

\subsection{Validation comparative}

\subsection{Benchmarks}

