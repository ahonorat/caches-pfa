\section{Description générale du produit}

\paragraph{}
Ce projet consiste en la création d'un simulateur de caches pour une architecture multi-c\oe ur (exclusivement \verb!x64!) dans le cadre d'un Projet de Fin d'Année de l'\textsc{Enseirb-MATMECA}, et sur le sujet proposé par Denis BARTHOU. Ce projet a été proposé dans un cadre pédagogique mais aussi un but utilitaire puisqu'il existe actuellement peu de programmes du même type.

\subsection{Au sujet de la simulation de caches}

\paragraph{}
La simulation des caches revient à parcourir des fichiers de traces d'exécutions d'un ou plusieurs programmes, afin de rejouer l'ensemble des accès et modifications de données effectués par les c\oe urs et leurs caches associés lorsqu'ils ont réellement exécuté ces programmes. Cela est à différencier d'un compteur hardware qui déterminerait l'ensemble des accès aux registres pendant l'exécution-même du programme à analyser. Par conséquent le simulateur est totalement indépendant de la machine sur laquelle les programmes ont été exécutés.   

\paragraph{}
Le résultat produit par le simulateur consiste en une analyse du jeu d'exécution, par le biais notamment du nombre de données ayant été lue/modifiée \emph{en mêm temps}\footnote{Techniquement il n'est pas possible de modifier deux données en même temps, car les caches communiquent entre eux par des bus serialisés, en pratique ces accès ne sont toutefois séparés que de quelques cycles ; cependant toute modification d'une donnée qui est lue dans un autre cache entraîne l'invalidation des autres copies de la donnée, ce qui est un processus assez couteux en temps.} par des c\oe urs différents, ce qui a ralenti l'exécution.

\subsection{Fonctionnalités nominales}

\paragraph{}
Le simulateur sera capable de rejouer et analyser une trace par c\oe ur (trace au format \texttt{MAQAO}) sur une architecture décrite par un fichier XML enrichi, initialement généré par \texttt{HWLoc}. Un certain nombre de configurations possibles ainsi que de métriques seront disponibles pour l'utilisateur. Aucune compilation supplémentaire des sources ne sera nécessaire après avoir choisi ces paramètres pour une exécution donnée puisqu'ils seront décrits dans des fichiers non-sources ou directement en argument du simultateur. 

\paragraph{}
Parmi les configurations possibles de la simulation, nous retrouverons : la politique de cohérence (pour chaque niveau de cache), la politique de remplacement d'une ligne (pour chaque cache), l'inclusivité des caches les uns par rapport aux autres (pour chaque niveau de cache), et la gestion de l'ordre d'éxécution des traces\footnote{L'ensemble des configurations possibles est présenté dans la rubrique \ref{politiques}.}.

\paragraph{}
La sortie du simulateur permettra de faire des analyses sur ce qui s'est produit dans le cache durant l'exécution d'une ou plusieurs traces. Cette sortie pourra être utilisée afin d'améliorer les performances d'accès à la mémoire d'un programme, et il sera possible de choisir la métrique à utiliser pour l'analyse courante\footnote{L'ensemble des métriques disponibles est présenté dans la rubrique \ref{métriques}.}. 

\subsection{Fonctionnalités annexes}

Si les délais le permettent, plusieurs options seront rajoutées : nous pensons notamment à deux ajouts concernant les métriques et la lecture de la trace. Pour les métriques, il s'agit de pouvoir suivre toutes les opérations effectuées sur une plage d'adresses donnée. Pour la lecture de la trace, il s'agit de l'optimisation du déroulement des boucles lors des demandes d'accès importantes à la mémoire, afin de ne pas parcourir la boucle en entier.

\subsection{Environnement de fonctionnement et contraintes}

Le simulateur ne sera pas utilisable pour toutes les configurations possibles, ni dans tous les environnemts existants. La contrainte principale, \texttt{MAQAO}\footnote{MAQAO n'est disponible que sous Linux, en version x64.}, nous a par exemple incité à se concentrer sur un déploiement sur des machines de type Linux uniquement. 

\subsubsection{Contraintes d'interfaçage}

Habituellement plusieurs outils sont utilisés pour l'étude d'une exécution parallèle, et les résultats de deux de ces outils (\texttt{MAQAO} et \texttt{HWLoc}) sont nécessaires au simulateur de caches. En effet \texttt{MAQAO} génère les traces d'exécutions à analyser, tandis que \texttt{HWLoc}
génère les fichiers de configurtion de l'architecture des caches. Il sera possible d'utiliser le simulateur en dehors de ces outils, mais la forme du fichier de configuration de l'architecture devra respecter celle d'\texttt{HWLoc}, tandis que la forme de la trace devra respecter celle de \texttt{MAQAO}.

\subsubsection{Systèmes d'exploitation compatibles}

Le simulateur sera uniquement développé pour les systèmes basés sur Linux. Bien qu'écrit en \texttt{C}, les outils utilisés ne fonctionnent que sur certaines architectures (notamment la lecture des fichiers). Par ailleurs les traces portent obligatoirement sur des adresses codées sur 64 bits mais il sera possible d'exécuter le simulateur sur une architecture \verb!x86!, cela risque toutefois de nuire aux performances du simulateur.

\subsubsection{Visualisation des résultats}

La visualisation des résultats (de même que le lancement du simulateur) se fera dans une console Linux, avec un format adapté à la métrique choisi. Aucune interface graphique n'est prévue, mais la génération d'un fichier en XML ou la réutilisation de l'interface de \texttt{HWLoc} est une option à long terme.
