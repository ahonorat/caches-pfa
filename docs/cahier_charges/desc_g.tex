\section{Description générale du produit}

\paragraph{}
Ce projet consiste en la création d'un simulateur de caches pour une architecture multi-c\oe ur. La simulations des caches revient à parcourir des fichiers de traces d'éxécutions d'un ou plusieurs programmes, afin de rejouer l'ensemble des accès et modifications de données effectuées par les processeurs et leurs caches associés lorsqu'ils ont réellement éxécuté ces programmes.

\subsection{Environnement de fonctionnement}

\paragraph{Contraintes d'interfaçage}
Habituellement plusieurs outils sont utilisés pour l'étude d'une éxécution parallèle, et les résultats de deux de ces outils (\textbf{MAQAO} et \textbf{HWLoc}) sont nécessaires au simulateur de caches. En effet \textbf{MAQAO} génère les traces d'éxécutions à analyser, tandis que \textbf{HWLoc}
génère les fichiers de configurtion de l'architecture des caches.

\subsection{Fonctionnalités nominales}

\subsection{Fonctionnalités annexes}
