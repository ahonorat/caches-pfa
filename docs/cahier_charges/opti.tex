\section{Optimisation des boucles}

\textsc{MAQAO} fournit des traces compressées, il peut donc être intéressant d'étudier plus précisément le comportement général des accès réalisés par un c{\oe}ur afin de détecter des similitudes. Dans tous les cas, ces optimisations ont des conséquences sur l'utilisation du simulateur, notamment sur sa rapidité, mais aussi sur sa précision.

\subsection{Amélioration des chargements de lignes}

Le premier schéma que nous proposons d'améliorer est le suivant:\\
\begin{lstlisting}
for i0 = 0 to 31999
  val 0x829c90 + 4*i0
endfor
\end{lstlisting}

Les interactions avec la mémoire (\emph{load}/\emph{store}) sont ici réalisées tous les $4$ octets pour un grand ensemble de valeurs. Par ailleurs, sur un architecture classique, de type \textit{Ivy Bridge} par exemple, la taille d'une ligne de cache est de $64$ octets.

Supposons qu'un thread réalise $16$ \emph{load} sur un ensemble de blocs mémoires, distants de $4$ octets, et que le premier bloc mémoire considéré soit un multiple de $64$. Il serait possible de faire le premier \emph{load}, afin de mettre à jour les données nécessaires. Par la suite les $15$ load restants feront des \emph{hits}. En réalisant, dans le simulateur, un nouveau \emph{load} et en enregistrant l'ensemble des modifications, le résultat final consiste simplement à multiplier par $15$ ces modifications.

Cette optimisation présente une limite : il faut effectuer en une seule fois les $16$ opérations pour considérer le résultat exact (sinon une autre trace aurait pu modifier un des données entre-temps). Il convient à l'utilisateur de vérifier avant d'effectuer la simulation avec cette optimisation que si le gain de temps du au plus faible nombre de \emph{load} compense la perte de temps à déterminer les optimisations possibles.

\subsection{Autres optimisations}

D'autres optimisations sont envisageables, notamment le non-déroulage des boucles, qui permet une exécution plus rapides des boucles mais provoque une imprécision sur les problèmes de cohérence qu'elles auraient pu provoquer. Dans ce cas l'utilisateur perd en précision pour gagner en temps de calcul.
