\chapter{Simulateur de caches : \textsc{Cassim}}

En vrac, non trié.

===> Rajouter une intro de partie rappelant la nécéssité d'un simulateur ? (par manque de connaissance/documentation dans le domaine, etc...)
===> Enchaîner ensuite sur le fait qu'on doit pouvoir choisir les paramètres de simulation (pour pouvoir simuler des architectures variés, et même expérimentaux ou pour obtenir le suivi de certaines données particulières)
===> Comment on a implémenté cette entre-guillemets "modularité" -- comment l'utilisateur peut choisir ses paramètres (script Lua pour l'entrelacement, XML c'est déjà fait, etc...)
===> D'autres trucs sur l'implém, par exemple préciser ce que ne fait pas le simulateur (calcul de bande passante, prise en compte du prefetching) ???
===> Exemple d'entrée/sortie, comparaison de la sortie avec ceux obtenus avec d'autres logiciels (compteur hard) + graphes en fonction de la taille de l'entrée (pour voir s'il n'y a pas de déviation lorsqu'on a des programmes plus complexes)

\section{Algorithmes de simulation}
Cette section entend présenter différents algorithmes utilisés par notre simulateur. Le but du simulateur est de produire des statistiques à partir d'une trace des blocs mémoires utilisés (\textit{load}/\textit{store}). Les données utilisées ne sont pas simulées. Le comportement du simulateur peut donc varier de la réalité, pourvu que le résultat final soit identique.

\subsection{Gestion de la cohérence}
La cohérence étant propre à un niveau de cache, elle peut être mise en place directement en considérant l'ensemble des caches de ce niveau. En réalité, les L1 ou les L2 n'intéreagissent pas toujours directement: s'ils ne sont pas reliés par un bus (cas du snooping), il faut passer par les niveaux haut dessus pour envoyer les messages. Cependant, la quantité de messages envoyés par cache pouvant être retrouvée à partir du nombre de \textit{misses} et de \textit{hits}, il n'est pas nécessaire de la calculer. \\

L'algorithme de cohérence (MSI, MESI, MOSI ou MOESI) est donc relativement facile à implémenter. Dans le cas du protocole MESI, il y a deux cas majeurs à gérer: \\
\begin{itemize}
\item Lorsqu'un cache fait un \textit{miss}, il parcours l'ensemble des autres caches pour savoir s'il doit mettre la donnée dans l'état E ou S. Dans le cas S, il modifie les données des autres caches pour quelles soient dans l'état S.
\item Quand un cache modifie une donnée qu'il avait dans l'état S, il invalide la donnée dans les autres caches.
\end{itemize}

\subsection{Interaction entre les différents niveaux de caches}

\subsection{Gestion des différentes modularités}

\section{Paramétrisation de l'architecture}

L'architecture à simuler peut être générée à partir de l'architecture réelle de l'utilisateur au moyen d'un fichier XML créé par le logiciel \emph{HWLOC}. Cependant l'utilisateur peut utiliser un fichier de paramétrisation spécifique à notre simulateur qui lui permet d'accéder à l'intégralité des paramètres d'architecture pris en compte.

\subsection{Entrée XML HWLOC}

\emph{HWLOC} est un logiciel libre sous liscence BSD-2. Il permet de générer un fichier XML qui décrit l'architecture de la machine utilisée (commande \verb?lstopo --of xml?). Il décrit notamment la structure arborescente des caches, et donne des informations essentielles pour chaque cache, comme sa taille, la taille de ses lignes et son associativité. 

\paragraph{}
Si l'utilisateur choisit un tel fichier en entrée comme décrivant son architecture, ce dernier sera parsé en un fichier de configuration de l'architecture personnalisé, comme décrit dans la section \ref{config}. Les paramètres non fournis par le fichier généré par \emph{HWLOC} prendront des valeurs par défaut, proches de celles des architectures \emph{intel} moderne. Notons que notre simulateur ne prend pas en compte les caches de niveau 1 dédiés aux instructions (L1i), qui sont décrits par \emph{HWLOC} mais ne seront pas présent dans le fichier personnalié.

\subsection{Fichier de configuration personnalisé}
\label{config}
Le fichier de configuration de l'architecture dédié à notre simulateur comprend tous les paramètres d'achitecture utilisables. Une fois généré à partir d'un fichier \emph{HWLOC}, il est possible de l'utiliser directement en entrée du simulateur, après avoir été modifié à la convenance de l'utilisateur.

\paragraph{}
Il s'agit d'un fichier XML qui contient 3 balises :
\begin{itemize}
  \item \textbf{Architecture} : donne le nom de l'architecture et du modèle de microprocesseur utilisé, ainsi que le nombre de niveaux de cache.
  \item \textbf{Level} : décrit un niveau de cache. Pour chaque profondeur, le protocole de cohérence, le type d'inclusivité, la présence ou non de \textit{snooping} et d'un \textit{directory manager}.
  \item \textbf{Cache} : décrit l'arborescence des caches. Pour chaque cache, sa profondeur, sa taille, la taille de ses lignes, son associtivité et son protocole de remplacement.
\end{itemize}

\subsubsection{Paramètres de la balise Architecture}

\begin{itemize}
  \item \textbf{name} : nom de l'architecture (ex : \textit{x86\_64})
  \item \textbf{CPU\_name} : modèle du microprocesseur
  \item \textbf{number\_levels} : nombre de niveaux de cache
\end{itemize}

\subsubsection{Paramètres de la balise Level}

\begin{itemize}
  \item \textbf{depth} : la profondeur du niveau. La valeur doit être cohérente avec le nombre de niveaux de l'architecture.
  \item \textbf{coherence\_protocol} : \verb?MESI? ou \verb?MSI?
  \item \textbf{type} : \verb?inclusive?, \verb?exclusive?, \verb?niio? (Non Inclusive, Inclusive Oriented) ou \verb?nieo? (Non Inclusive, Exclusive Oriented)
  \item \textbf{snooping} : \verb?true? ou \verb?false?
  \item \textbf{directory\_manager} : \verb?true? ou \verb?false?
\end{itemize}

\subsubsection{Paramètres de la balise Cache}
Ces balises doivent former l'arborescence des caches.

\begin{itemize}
  \item \textbf{depth} : la profondeur du cache.
  \item \textbf{cache\_size} : taille du cache (en octets)
  \item \textbf{cache\_linesize} : taille d'une ligne de cache (en octets)
  \item \textbf{cache\_associativity} : associativité du cache
  \item \textbf{remplacement\_protocol} : \verb?FIFO?, \verb?LRU? ou \verb?LFU?
\end{itemize}

\subsubsection{Exemple de fichier}

%TODO : mettre un exemple de fichier à jour

